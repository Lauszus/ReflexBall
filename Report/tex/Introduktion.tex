\chapter{Introduktion}
Denne rapport dokumenterer de overvejelser vi har gjort os i forhold til design og struktur af Reflex Ball spillet. \\
Programmeringsprojektet har overordnet set været inddelt i to faser.\\

Første del, der inkluderede de fire første arbejdsdage, blev brugt på at lave håndregningsøvelser omhandlende manipulation af hexadecimal- og bitrepresenterede tal og flere programmerings-øvelser i C omhandlende hexadecimal- og bitmanipulation samt ansi-koder, driverhåndtering, vektorregning og styring a LED-display. Formålet med disse øvelser var samlet set at blive bedre til at håndtere hexadecimal- og bitrepresenterede tal, og desuden at lære om/genopfriske pointers, headers og opsætning af hardware-drivere til microcontrolleren i C.\\

Anden del af projektet, der inkluderede de sidste ni arbejdsdage, har vi brugt på først at designe Reflex Ball-spillet ud fra de i opgaven specificerede krav og derefter implementere udvidelser både på hardware- og softwareniveau, sideløbende med vi hele tiden har videreudviklet spillets 'engine', så spillet bruger mindre RAM og kører så fejlfrit som muligt, samtidig med at koden gradvis er blevet gjort mere fleksibel (ingen hardcode-stil) og mere overskuelig.\\
Som sidste led af anden del af projektet, har vi skrevet denne rapport.
